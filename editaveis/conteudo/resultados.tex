\chapter[Resultados]{Resultados}
\label{cp:resultados}

O Grata foi desenvolvido utilizando o \textit{Python Rest} como \textit{Backend} e o \textit{React} como \textit{Frontend}. Para facilitar futuras evoluções foi utilizado o \textit{Dcoker} e o \textit{Deploy} foi realizado na plataforma \textit{Heroku}. Tudo desenvolvido para este projeto pode ser encontrado nos repositórios:
\begin{itemize}
    \item Documentação Grata: \url{https://github.com/FGAProjects/TCC};
    \item Frontend: \url{https://github.com/MrVictor42/Grata-Frontend-v2};
    \item Backend: \url{https://github.com/MrVictor42/Grata-Backend-v2}.
\end{itemize}

O links dos deploys da ferramenta:
\begin{itemize}
    \item Backend: \url{https://api-grata.herokuapp.com/}
    \item Frontend: \url{https://projeto-grata.herokuapp.com/}
\end{itemize}

Mais informações referentes ao projeto, podem ser vistas no tópico \ref{sec:manual}.