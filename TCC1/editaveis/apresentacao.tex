
O presente relatório faz parte das atividades a serem desenvolvidas na disciplina de Projeto Integrador 1, do segundo semestre de 2015, onde seis grupos foram sorteados para o desenvolvimento de seis projetos definidos pelos professores da disciplina. O progresso das atividades são avaliadas em três pontos de controle, onde há a apresentação do conteúdo desenvolvido.

O grupo 1 ficou responsável por elaborar o projeto de um sistema de monitoramento do campus da UnB-FGA baseado num balão cativo. As atividades concluídas até o presente momento incluem os objetivos propostos para o Ponto de Controle 1, definição do escopo, escolha da metodologia a ser utilizada, cronograma de atividades, Estrutura Analítica de Projeto e e objetivos propostos para o Ponto de Controle 2, ou seja, pesquisas baseadas nos requisitos elencados, modelagem 3D das estruturas do balão, diagramas esquemáticos e comparativos entre as propostas de solução e definição da mesma.

Dessa forma, como parte da terceira etapa do projeto, o Relatório 3 desenvolve  as atividades propostas pelo terceiro Ponto de Controle, que é,além do desenvolvimento de modelos esquemáticos dos processos das grandes áreas de pesquisas, cálculos de consumo energético, empuxo, forças atuantes no balão, volume da bexiga e de gás, também são os valores de custo dos equipamentos em cada grande área de pesquisa e cálculo da viabilidade econômica do projeto. 