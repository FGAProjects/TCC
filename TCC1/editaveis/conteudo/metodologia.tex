\chapter[Metodologia]{Metodologia}
\label{cp:metodologia}

\section{Metologia de Desenvolvimento}

Por conta de possuir um maior conhecimento sobre a metodologia e pela proposta em entregas mais frequentes em períodos menores, foi escolhida a metodologia ágil para o desenvolvimento do \textit{software} juntamente com algumas práticas do \textit{Scrum}. 

\subsection{Scrum}

Após analisar os tipos de gerenciamento de projetos no tópico \ref{sec:gerenciamento_de_projetos}, foi escolhido a metodologia ágil com as vantagens do \textit{Scrum}, explicitadas no tópico \ref{sec:scrum} como metodologia de desenvolvimento deste TCC. Uma das principais vantagens do \textit{Scrum} é a adaptação dele aos projetos em que os processos de desenvolvimento não sejam tão rigorosos e o processo desse \textit{framework} será adaptado neste projeto.

\subsubsection{Papéis}

Este projeto será desenvolvido de forma individual, os papéis do \textit{framework} \textit{Scrum} foram distribuídos de forma com que cada ator tenha a seguinte responsabilidade: o responsável pelo desenvolvido do sistema, Victor Mota, assume os papéis de Time de Desenvolvimento e \textit{Scrum Master}. O papel de \textit{Product Owner} será assumido por Pedro Marques, servidor do Senado Federal e do NMIL, que é o setor de caso de estudo deste trabalho. 

\subsubsection{Sprints}

Para este projeto, as \textit{Sprints} foram definidas com duração máxima de duas semanas. Assim como define no \textit{Scrum}, as \textit{Sprints} devem ter as atividades de planejamento e revisão da mesma, para que seja constatado se o que foi planejado foi entregue ao longo das duas semanas.

\begin{itemize}
    \item \textit{Sprint Planning}: Esta atividade é realizada no primeiro dia de \textit{Sprint} e é nela que são selecionados os itens do \textit{Product Backlog} que serão desenvolvidos ao longo da \textit{Sprint};
    \item \textit{Sprint Review}: Esta atividade é realizada no último dia de \textit{Sprint} e é nela são discutidas com \textit{Product Owner} as histórias de usuário desenvolvidas ao longo da \textit{Sprint}.   
\end{itemize}