\chapter[Fundamentação Teórica]{Fundamentação Teórica}
\label{cp:fundamentacao}

\section{Gerenciamento de Projetos}

\subsection{Modelo Tradicional}

Uma metodologia de gerenciamento de projetos no modelo tradicional, de acordo com \cite{kerzner} é o alcance da excelência no gerenciamento de projetos se torna impossível sem um processo repetitivo que possa ser utilizado em cada projeto.

No modelo tradicional, um dos mais modelos mais utilizados é o RUP \textit{(Rational Unified Process)}. O RUP oferece uma metodologia responsável por responder questões como boas práticas para o gerenciamento de projetos, com o objetivo de estruturar e formatar os processos associados às atividades que envolvem a tecnologia de informação. 

As 4 fases principais do \cite{rup} pode ser vista na figura \ref{img:fases_do_rup}:

\begin{figure}[H]
	\centering
	\includegraphics[width=1.0\textwidth]{figuras/fases_rup.jpg}
	\caption{Fases do RUP. Fonte: \cite{rup}.}
	\label{img:fases_do_rup}
\end{figure}

O desenvolvimento do plano de gerenciamento do projeto é uma atividade iterativa ao longo do ciclo de vida do projeto, sempre pronto para melhoria contínua e permitindo à equipes do projeto definir e trabalhar com maior nível de detalhes. De acordo com o \cite{pmbok}, as fases do \cite{rup} são sobrepostas, ou seja, o início de uma fase é ao término de uma outra, isso leva a algumas atividades ocorrerem de forma paralela. A maneira como este tipo de projeto aumenta os riscos, retrabalhos, e exigir recursos adicionais para permitir as atividades em paralelo, como mostrado na figura \ref{img:fases_do_rup}.

Nesta perspectiva, se tem o papel do gerente de projeto como um líder responsável por liderar a equipe para alcançar os objetivos previstos no planejamento do projeto. Entre as funções destes líderes se tem:

\begin{itemize}
    \item Conhecimento acerca do gerenciamento de projetos;
    \item Desempenho para aplicar seus conhecimentos na prática;
    \item Comportamento pessoal de liderança, atingindo objetivos e equilibrando restrições.
\end{itemize}

Este tipo de gerenciamento de projeto é mais utilizado em empresas já consolidadas, de ramo mais formal, que possui mais burocracia em seus projetos e por tanto maior rigor de documentação e de liderança dos gerentes de projeto. Essa hierarquia pode ser vista na figura \ref{img:gerencia_de_projetos_tradicional}.

\begin{figure}[H]
	\centering
	\includegraphics[width=1.0\textwidth]{figuras/gerencia_de_projeto.jpg}
	\caption{Hierarquia em projetos tradicionais. Fonte: \cite{gerentes_tradicionais}.}
	\label{img:gerencia_de_projetos_tradicional}
\end{figure}

\subsection{Modelo Ágil}

Em contraposição ao modelo tradicional, surte o manifesto ágil como uma reação contra o processo burocrático presente no modelo tradicional, que possuem por característica atividades sequências em modelo cascata. Segundo a \cite{chaos} apenas 16,2\% dos projetos entregues por companhias americanas foram entregues respeitando prazos, custos previamente acordados e objetivos determinados. Segundo a própria \cite{chaos}, as principais causas destes problemas estavam relacionadas com o modelo sequencial tradicional.

O modelo ágil, segundo \cite{soares}, ela deve primeiro aceitar as mudanças em vez de tentar prevê-las, agir de maneira rápida sabendo receber, avaliar e responder como elas devem ser respondidas. As principais características da metodologia ágil são:

\begin{itemize}
	\item Desenvolvimento iterativo e incremental;
	\item Comunicação;
	\item Documentação extensiva; 
\end{itemize}

Em 2001, membros da comunidade de \textit{software} se reuniram e criaram o \cite{agile_manifest}. O objetivo deste manifesto é utilizar as melhores práticas observadas em projetos anteriores que obtiveram sucessos.

Os principais conceitos do manifesto ágil são:

\begin{itemize}
	\item Indivíduos e interações ao invés de processos e ferramentas;
	\item \textit{Software} executável ao invés de documentação;
	\item Colaboração do cliente ao invés de negociação de contratos;
	\item Resposta rápida a mudanças ao invés de seguir planos pré-estabelecidos.
\end{itemize}

Uma das boas práticas adotadas ao modelo ágil é o SCRUM. O SCRUM, se refere ao jogo \textit{Rugby}, que é a ação dos jogadores se organizarem em círculo para planejar a próxima jogada. Um dos principais pontos de vista do SCRUM é mostrar um projeto com pequenos ciclos, aumentando as iterações entre os participantes, mas com visão a longo prazo.

O ciclo de vida pode ser visto na figura \ref{img:ciclo_de_vida_scrum}:

\begin{figure}[H]
	\centering
	\includegraphics[width=1.0\textwidth]{figuras/ciclo_de_vida_scrum.png}
	\caption{Ciclo de Vida SCRUM. Fonte: \cite{scrum}.}
	\label{img:ciclo_de_vida_scrum}
\end{figure}

Como visto na figura \ref{img:ciclo_de_vida_scrum}, o SCRUM é um ciclo progressivo de várias iterações bem definidas, denominadas \textit{Sprints}. As \textit{Sprints} podem ter duração de uma a quatro semanas. Antes de cada \textit{Sprint}, deve ser realizada a reunião de planejamento da \textit{Sprint}, chamada de \textit{Sprint Planning Meeting}, na qual os desenvolvedores tem contato com o \textit{Product Owner}, que possui o dever de priorizar as atividades, seleciona-las e estimar as tarefas que a equipe pode desenvolver na próxima \textit{Sprint}.

Com o objetivo de saber o progresso de cada equipe dentro da \textit{Sprint}, ocorrem as reuniões diárias, denominadas \textit{Daily Meetings}, que tem duração de no máximo 15 minutos e ocorrem com todos os participantes em pé, respondendo perguntas como: "O que você fez ontem?", "O que você fez hoje?" e "O que você vai fazer amanhã?". 

Ao final de uma \textit{Sprint} é feita uma análise gráfico do progresso do projeto atráves do \textit{Sprint Backlog} durante a \textit{Sprint Review}. Após a \textit{Sprint Review} ocorre a \textit{Sprint Restrospective} que é a análise de experiências que ocorreram durante a \textit{Sprint} sejam boas ou não a fim de melhora-las.

Segundo \cite{fowler}, as equipes devem possuir um quadro para registro das atividades, denominado \textit{Kanban}. O \textit{Kanban} possui o objetivo de auxiliar as equipes em relação ao progresso da \textit{Sprint}, esse quadro pode ser dividido em 4 fases:

\begin{itemize}
	\item Para fazer;
	\item Em andamento (com o nome do responsável pela atividade);
	\item Em revisão;
	\item Feito.
\end{itemize}

\begin{figure}[H]
	\centering
	\includegraphics[width=1.0\textwidth]{figuras/kanban.png}
	\caption{Quadro Kanban. Fonte: \cite{kanban}.}
	\label{img:kanban}
\end{figure}

No modelo ágil os requisitos dos clientes podem ser mudados a qualquer momento, e o time de gerência e desenvolvimento devem estar preparados para conversar com o cliente a fim de resolver as alterações de requisitos da melhor maneira possível. Este tipo de pensamento no modelo tradicional é mais difícil de acontecer, pois ao observar a figura \ref{img:fases_do_rup}, é possível notar ao iniciar uma fase, essa mesma fase não é retornada mais tarde, ou seja, no modelo tradicional uma troca de requisitos pode levar ao reinicio do projeto.

Este modelo é mais focado para empresas emergentes, que não são muito rigorosas em seus processos e aceita que mudanças nos requisitos ou na visão do produto são sempre bem vindas, desde que melhore o projeto final. 

% Tendo como premissas estes problemas, e a vontade de facilitar a operação, foram levantados a partir de estudos e dos conhecimentos adquiridos no curso de \imprimircurso uma solução que ajude gerentes e líderes de reuniões nas empresas a desenvolver encontros que sejam mais diretos e objetivos a fim de alcançar seus objetivos. Como estudo de caso foi escolhido o NMIL (Núcleo de Modernização da Informação Legislativa), um setor localizado no Senado Federal Brasileiro.

% O Senado Federal é uma instituição de âmbito nacional no Brasil, e dentre suas secretárias e setores, se tem o NMIL, o caso de estudo deste trabalho. Neste setor são realizadas reuniões quase que diariamente e a partir destes encontros podem ser gerados projetos para o próprio setor, como também para outras secretárias, então fica a cargo do NMIL realizar toda a tramitação e registrar todas as etapas. Essas informações são gravadas em papeis por vezes perdidos.

% O Sistema GRATA, vem oferecer a solução prática para a melhoria do controle das informações e qualidade dos serviços. Tendo como a principal funcionalidade o registro das Atas de reuniões de forma simples e intuitiva, tanto para quem gerencia como para quem participa. Além da automação dos processos essenciais da organização, o sistema fornece relatórios gerenciais e analíticos, que podem ser usados para identificação de pontos de melhoria ou até mesmo para dar visibilidade a questões específicas.

% \section{Reuniões Improdutivas e Suas Consequências}

% Na era do conhecimento em que estamos inseridos, reuniões estão cada vez mais presentes em organizações e segundo \cite{allen2016} essas reuniões já ocupam em média 15\% do tempo coletivo da organização. Contudo reuniões mal administradas podem levar ao desperdício de recursos da empresa, mas também a sensação dos participantes que a reunião ainda não terminou.

% O professor \cite{macleod} estima que 30\% a 60\% do tempo gasto com reuniões é desperdiçado. Gerentes podem passar por volta de 11 horas semanais com reuniões mal sucedidas, completando quase um total de 35 dias úteis ao ano. 71\% dos gerentes que foram pesquisados indicam que reuniões ineficazes os impedem de completar funções básicas em seus trabalhos, segundo \cite{perlow}.

% \textbf{COLOCAR MAIS SOBRE PROBLEMAS NO TRABALHO QUE NÃO SEJA APENAS O CUSTO}

% Uma das consequências de serem notadas em uma reunião ineficaz além de participantes dispersos e perdidos, é o custo. Estima-se que empresas gastam em média US \$ 37 bilhões anualmente em reuniões \cite{baer}. O custo real desses encontros impulsionou a \cite{harvard} a criar uma calculadora que ajuda gerentes a calcularem o verdadeiro custo de um encontro. 