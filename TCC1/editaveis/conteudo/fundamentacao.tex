\chapter[Fundamentação Teórica]{Fundamentação Teórica}
\label{cp:fundamentacao}

\section{Reuniões Tradicionais vs Reuniões no Modelo Ágil}

\subsection{Reuniões Tradicionais}

\subsection{Reuniões Ágeis}

% Tendo como premissas estes problemas, e a vontade de facilitar a operação, foram levantados a partir de estudos e dos conhecimentos adquiridos no curso de \imprimircurso uma solução que ajude gerentes e líderes de reuniões nas empresas a desenvolver encontros que sejam mais diretos e objetivos a fim de alcançar seus objetivos. Como estudo de caso foi escolhido o NMIL (Núcleo de Modernização da Informação Legislativa), um setor localizado no Senado Federal Brasileiro.

% O Senado Federal é uma instituição de âmbito nacional no Brasil, e dentre suas secretárias e setores, se tem o NMIL, o caso de estudo deste trabalho. Neste setor são realizadas reuniões quase que diariamente e a partir destes encontros podem ser gerados projetos para o próprio setor, como também para outras secretárias, então fica a cargo do NMIL realizar toda a tramitação e registrar todas as etapas. Essas informações são gravadas em papeis por vezes perdidos.

% O Sistema GRATA, vem oferecer a solução prática para a melhoria do controle das informações e qualidade dos serviços. Tendo como a principal funcionalidade o registro das Atas de reuniões de forma simples e intuitiva, tanto para quem gerencia como para quem participa. Além da automação dos processos essenciais da organização, o sistema fornece relatórios gerenciais e analíticos, que podem ser usados para identificação de pontos de melhoria ou até mesmo para dar visibilidade a questões específicas.

% \section{Reuniões Improdutivas e Suas Consequências}

% Na era do conhecimento em que estamos inseridos, reuniões estão cada vez mais presentes em organizações e segundo \cite{allen2016} essas reuniões já ocupam em média 15\% do tempo coletivo da organização. Contudo reuniões mal administradas podem levar ao desperdício de recursos da empresa, mas também a sensação dos participantes que a reunião ainda não terminou.

% O professor \cite{macleod} estima que 30\% a 60\% do tempo gasto com reuniões é desperdiçado. Gerentes podem passar por volta de 11 horas semanais com reuniões mal sucedidas, completando quase um total de 35 dias úteis ao ano. 71\% dos gerentes que foram pesquisados indicam que reuniões ineficazes os impedem de completar funções básicas em seus trabalhos, segundo \cite{perlow}.

% \textbf{COLOCAR MAIS SOBRE PROBLEMAS NO TRABALHO QUE NÃO SEJA APENAS O CUSTO}

% Uma das consequências de serem notadas em uma reunião ineficaz além de participantes dispersos e perdidos, é o custo. Estima-se que empresas gastam em média US \$ 37 bilhões anualmente em reuniões \cite{baer}. O custo real desses encontros impulsionou a \cite{harvard} a criar uma calculadora que ajuda gerentes a calcularem o verdadeiro custo de um encontro. 