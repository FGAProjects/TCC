\chapter[Fundamentação Teórica]{Fundamentação Teórica}
\label{cp:fundamentacao}

\textbf{AQUI VAI FICAR UM PEQUENO RESUMO DA FUNDAMENTAÇÃO TEÓRICA}

\section{Reuniões Improdutivas e Suas Consequências}

Na era do conhecimento em que estamos inseridos, reuniões estão cada vez mais presentes em organizações e segundo \cite{allen2016} essas reuniões já ocupam em média 15\% do tempo coletivo da organização. Contudo reuniões mal administradas podem levar ao desperdício de recursos da empresa, mas também a sensação dos participantes que a reunião ainda não terminou.

O professor \cite{macleod} estima que 30\% a 60\% do tempo gasto com reuniões é desperdiçado. Gerentes podem passar por volta de 11 horas semanais com reuniões mal sucedidas, completando quase um total de 35 dias úteis ao ano. 71\% dos gerentes que foram pesquisados indicam que reuniões ineficazes os impedem de completar funções básicas em seus trabalhos, segundo \cite{perlow}.

\textbf{COLOCAR MAIS SOBRE PROBLEMAS NO TRABALHO QUE NÃO SEJA APENAS O CUSTO}

Uma das consequências de serem notadas em uma reunião ineficaz além de participantes dispersos e perdidos, é o custo. Estima-se que empresas gastam em média US \$ 37 bilhões anualmente em reuniões \cite{baer}. O custo real desses encontros impulsionou a \cite{harvard} a criar uma calculadora que ajuda gerentes a calcularem o verdadeiro custo de um encontro. 