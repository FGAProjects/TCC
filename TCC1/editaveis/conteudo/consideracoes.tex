\chapter{Considerações Finais}
\label{cp:consideracoes}

Ao analisar estudos dos autores presentes neste projeto, além de ferramentas que auxiliem os gerentes de projetos no desenvolvimento de seus encontros, é possível observar que tanto a literatura quanto o mercado estão em busca de soluções que estão gerando tantos gastos quanto a diminuinção da satisfação dos funcionários na empresa. Esses dois fatores combinados, estão levando a criação de estudos e \textit{softwares} que diminuia os custos empresariais e otimize o processo de desenvolvimento dos projetos.

As pesquisas realizadas na introdução e sobretudo no referencial teórico deu um caminho maior ao autor deste projeto sobre estudos e \textit{softwares} que tentam resolver os problemas de gerenciamentos de reuniões improdutivas. A literatura deu o caminho e lacunas enfrentadas dos gerentes e os \textit{softwares} em saber os pontos fortes e fracos deles dos sistemas em vigor no mercado. Assim, será possível propor uma ferramenta de gerenciamento de reuniões que preencha as lacunas dos \textit{softwares} e que possa ser utilizado por gerentes de qualquer empresa de forma gratuita.

Como trabalho futuro, todo o aprendizado obtido durante o desenvolvimento deste trabalho será utilizado uma ferramenta gratuita, que permita aos gerentes de projetos criar suas reuniões, sabendo exatamente o que está acontecendo no projeto, e o que os outros participantes da reunião pensam sobre a reunião e projetos.