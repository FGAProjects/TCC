\chapter{Considerações Finais}
\label{cp:consideracoes}

Ao analisar estudos dos autores presentes neste projeto, além de ferramentas que auxiliem os gerentes de projetos no desenvolvimento de seus encontros, é possível observar que tanto a literatura quanto o mercado estão em busca de soluções que estão gerando tantos gastos quanto a diminuinção na insatisfação dos funcionários na empresa em relação a reuniões ineficazes. Esses dois fatores combinados, estão levando a criação de estudos e \textit{softwares} que diminuia os custos empresariais e otimize o processo de desenvolvimento dos projetos.

As pesquisas realizadas na introdução e sobretudo no referencial teórico deu um caminho maior ao autor deste projeto sobre estudos e \textit{softwares} que tentam resolver os problemas na condução de reuniões. A literatura mostrou algumas lacunas enfrentadas pelos gerentes e os \textit{softwares} similares em saber o que os usuários comentam sobre os pontos fortes e fracos desses sistemas em vigor no mercado. Assim, será possível propor uma ferramenta de gerenciamento de reuniões que preencha as lacunas dos \textit{softwares} e que possa ser utilizado por gerentes de qualquer empresa de forma gratuita.

Após passar por toda a fase de elicitação dos requisitos e desenvolvimento do \textit{software}, a resposta para a pergunta de pesquisa, levantada no tópico \ref{sec:problema_de_pesquisa}, se "\textit{É possível desenvolver um sistema que auxilie a condução de reuniões em organizações?}", a resposta é o encerramento deste projeto como resultado de muito trabalho e dedicação. 

Como trabalhos futuros, após a utilização da ferramenta, podem ser desenvolvidas novas funcionalidades, como uma maior personalização do perfil de usuário, novos tipos de usuários e formas diferentes de avaliações de reuniões.