\chapter[Introdução]{Introdução}
\label{cp:introducao}

Um dos instrumentos mais fundamentais e que são cada vez mais crescentes na vida organizacional de uma empresa são as reuniões. Segundo \cite{allen2016}, uma instituição utiliza em média 15\% do tempo coletivo da organização com reuniões.

\section{Contexto}
\label{sec:contexto}

Encontros institucionais que são improdutivas e sem sentido, é o que \cite{davidgrady} chama de "Síndrome de Aceitação Sem Sentido"(MAS). David define o MAS como "um reflexo involuntário em que uma pessoa aceita um convite de reunião sem sequer saber o porquê. Uma doença comum entre o escritório e trabalhadores em todo mundo". Atividades que deveriam ser simples e rápidas se tornam complicadas com várias reuniões para a execução completa delas. Reuniões não são nenhum ponto de prazer dentro de uma instituição, contudo se os próprios funcionários não conseguem visualizar o sentido da reunião e os tópicos abordados, mostra que a empresa como um todo está fadada ao fracasso.

\section{Justificativa}
\label{sec:justificativa}

As reuniões são criadas para promover o compartilhamento de informações, melhorar a tomada de decisões, promover a resolução de problemas, construir a coesão da equipe e reforçar a cultura organizacional \cite{leach}. Por se tratar de um encontro dentro da empresa, segundo \cite{leach}, as reuniões podem gerar emoções tanto positivas quanto negativas dentre os participantes da reunião. É possível sair de uma reunião sentindo-se energizado e inspirado, ou afastar-se de uma reunião sentindo-se esgotado e desmoralizado.  

\cite{macleod} estimou que entre 30\% a 60\% do tempo gasto em uma reunião é desperdiçado. Um gerente pode passar onze horas por semanas em reuniões e metade desse tempo é improdutivo. Ao realizar uma pesquisa sobre a produtividade das reuniões, \cite{perlow} constatou que 65\% dos gerentes entrevistados indicaram que reuniões os impedem de realizar suas próprias atividades no trabalho e 71\% alegam que as reuniões são improdutivas e ineficazes. Uma reunião pode render comentários positivos e negativos, contudo segundo \cite{leach}, os comentários negativos são os mais pertinentes e são relacionados a estrutura da reunião. Problemas como falta de planejamento, informações de baixa relevância e impacto pouco claro de participação são os fatores que mais compõem negativamente uma reunião.

\cite{rogelberg} examinou as reuniões a partir de duas teorias: capacidade atencional e teoria da ação. Ao aplicar essas práticas, foi constatado que a fadiga diária e a carga de trabalho subjetiva estão relacionadas diretamente as reuniões atendidas. O estudo sugere ainda que a frequência de reuniões é mais importante do que o tempo gasto em reuniões ao longo do dia. Ainda segundo \cite{rogelberg}, a "natureza disruptiva dos resultados das reuniões na drenagem recursos emocionais ou mentais e fadiga subsequente". A conclusão do estudo foi que tanto a qualidade quanto a quantidade das reuniões são importantes para o bem-estar do funcionário. 

As reuniões por mais importantes que sejam, não apresentam de fato o verdadeiro sentido que elas possuem. Como apresentado nos tópicos anteriores, estudos nessa área apresentam diversos problemas nas reuniões  com níveis de improdutividade elevados, contudo ao analisar esses estudos é possivel elaborar uma solução computacional que auxilie gerentes e funcionários a não gastarem tanto tempo em reuniões e atingir um maior nível de produtividade. A proposta computacional visa aumentar o engajamento dos participantes, aumentar no planejamento das reuniões, informações com maiores níveis de relevância e por fim diminuir os altos níveis de improdutividade apresentados pelos autores nos tópicos anteriores.

Ao realizar uma pesquisa de mercado, foi encontrado quatro \textit{softwares} semelhantes ao proposto nesse projeto, sendo eles: 

\begin{itemize}
    \item \textit{Meeting} (gestão de projetos)
    \item Qualiex
    \item Eata
    \item \textit{Google calendar}
\end{itemize}

Desses quatro \textit{softwares}, os três primeiros são pagos, contendo uma versão trial de 30 dias. O \textit{Google calendar} é um \textit{software} gratuito, contudo não é tão completo os outros três.  

\section{Problema de Pesquisa}
\label{sec:problema_de_pesquisa}

O crescente problema com reuniões mal gerenciadas seja por gerentes não capacitados, ou por falta da especificação prévia dos tópicos a serem abordados, levam diretamente a reuniões mal sucessididas e com isso desperdício de tempo e dinheiro. Estima-se que empresas gastam em média US \$ 37 bilhões anualmente em reuniões \cite{baer}. O custo real desses encontros impulsionou a \cite{harvard} a criar uma calculadora que ajuda gerentes a calcularem o verdadeiro custo de um encontro.

Nessa viés e utilizando a justificativa desenvolvida no tópico \ref{sec:justificativa}, é possível se ter o problema de pesquisa. A problemática a ser resolvida neste projeto é: \textit{Como desenvolver um sistema para auxiliar a condução de reuniões em organizações?}

\section{Objetivos}
\label{sec:objetivos}

\subsection{Objetivos Gerais}
\label{sec:objetivos_gerais}

O objetivo do projeto é propor uma solução computacional que auxilie gerentes de qualquer empresa a gerenciar suas reuniões, possuindo um controle sobre os documentos gerados e conseguir tornar suas reuniões mais produtivas. Com o suporte de um \textit{software} gratuito e de código aberto para automatizar trabalhos manuais, tornando os processos mais ágeis e com armazenamentos seguros.

\subsection{Objetivos Específicos}
\label{sec:objetivos_especificos}

\begin{itemize}
    \item Criar um \textit{software} que auxilie gerentes e líderes a terem reuniões mais objetivas;
    \item implementar mecanismos que permita o controle gerencial das reuniões;
    \item permitir o controle de todas as informações provindas das reuniões;
    \item desenvolver um \textit{software} de código aberto, gratuito e que atenda a demanda de gerenciar as reuniões;
    \item desenvolver um \textit{software} que possa ser utilizado em qualquer empresa.
\end{itemize}