\chapter[Introdução]{Introdução}

Reuniões é um dos instrumentos mais fundamentais e cada vez mais crescentes na vida organizacional de uma empresa. Segundo \cite{allen2016}, se gasta até 15\% do tempo coletivo da organização com reuniões. O tempo gasto e seu retorno com o aproveitamento das reuniões são cargos do líder ou do gerente do setor, contudo reuniões não podem ser otimizadas? Não podem ser mais objetivas e focar nos que são os tópicos da reunião?

Encontros institucionais que não são produtivas e sem sentido, é o que \cite{davidgrady} chama de "Síndrome de Aceitação Sem Sentido"(MAS). David define o MAS como "um reflexo involuntário em que uma pessoa aceita um convite de reunião sem sequer saber o porquê. Uma doença comum entre o escritório e trabalhadores em todo mundo". Atividades que deveriam ser simples e rápidas se tornam complicadas com várias reuniões para a execução completa delas. Reuniões não são nenhum ponto de prazer entre dentro de uma instituição, contudo se os próprios funcionários não conseguem ver o sentido da reunião e os tópicos abordados, mostra que a empresa como um todo está fadada ao fracasso. 

Nesse viés, reuniões é um dos meios usados para programar uma atividade, reunir pessoas em busca de uma solução relacionada ao problema X, e vários encontros sem tópicos e objetivos específicos, sem feedbacks aos gerentes e sem atas sobre o que foi discutido, infelizmente é algo presentes em instituições iniciais. Problemas como estes podem ser pela falta de investimento disponível para aplicação na área tecnológica ou até mesmo pela priorização de outras necessidades.

Tendo como premissas estes problemas, e a vontade de facilitar a operação, foram levantados a partir de estudos e dos conhecimentos adquiridos no curso de \imprimircurso uma solução que ajude gerentes e líderes de reuniões nas empresas a desenvolver encontros que sejam mais diretos e objetivos a fim de alcançar seus objetivos. Como estudo de caso foi escolhido o NMIL (Núcleo de Modernização da Informação Legislativa), um setor localizado no Senado Federal Brasileiro.

O Senado Federal é uma instituição de âmbito nacional no Brasil, e dentre suas secretárias e setores, se tem o NMIL, o caso de estudo deste trabalho. Neste setor são realizadas reuniões quase que diariamente e a partir destes encontros podem ser gerados projetos para o próprio setor, como também para outras secretárias, então fica a cargo do NMIL realizar toda a tramitação e registrar todas as etapas. Essas informações são gravadas em papeis por vezes perdidos.

O Sistema GRATA, vem oferecer a solução prática para a melhoria do controle das informações e qualidade dos serviços. Tendo como a principal funcionalidade o registro das Atas de reuniões de forma simples e intuitiva, tanto para quem gerencia como para quem participa. Além da automação dos processos essenciais da organização, o sistema fornece relatórios gerenciais e analíticos, que podem ser usados para identificação de pontos de melhoria ou até mesmo para dar visibilidade a questões específicas.

\section{Estrutura do TCC}

\textbf{DEFINIR AS ESTRUTURA DO TCC AQUI}