\chapter[Introdução]{Introdução}
\label{cp:introducao}

\section{Apresentação do Tema}
\label{sec:apresentacao}

Um dos instrumentos mais fundamentais e que são cada vez mais crescentes na vida organizacional de uma empresa são as reuniões. Segundo \cite{allen2016}, uma instituição utiliza em média 15\% do tempo coletivo da organização com reuniões.

Encontros institucionais que são improdutivas e sem sentido, é o que \cite{davidgrady} chama de "Síndrome de Aceitação Sem Sentido"(MAS). David define o MAS como "um reflexo involuntário em que uma pessoa aceita um convite de reunião sem sequer saber o porquê. Uma doença comum entre o escritório e trabalhadores em todo mundo". Atividades que deveriam ser simples e rápidas se tornam complicadas com várias reuniões para a execução completa delas. Reuniões não são nenhum ponto de prazer dentro de uma instituição, contudo se os próprios funcionários não conseguem visualizar o sentido da reunião e os tópicos abordados, mostra que a empresa como um todo está fadada ao fracasso.

\section{Justificativa}
\label{sec:justificativa}


\section{Problema de Pesquisa}
\label{sec:problema_de_pesquisa}

O crescente problema com reuniões mal gerenciadas seja por gerentes não capacitados, ou por falta da especificação prévia dos tópicos a serem abordados, levam diretamente a reuniões mal sucessididas e com isso desperdício de tempo e dinheiro. Estima-se que empresas gastam em média US \$ 37 bilhões anualmente em reuniões \cite{baer}. O custo real desses encontros impulsionou a \cite{harvard} a criar uma calculadora que ajuda gerentes a calcularem o verdadeiro custo de um encontro.

Nessa viés e utilizando a justificativa desenvolvida no tópico \ref{sec:justificativa}, é possível se ter o problema de pesquisa. A problemática a ser resolvida neste projeto é: \textit{Como desenvolver um sistema para auxiliar a condução de reuniões em organizações?}

\section{Objetivos}
\label{sec:objetivos}

\subsection{Objetivos Gerais}
\label{sec:objetivos_gerais}

Os objetivos do projeto são de propor novos processos de reuniões e gerenciamento dos documentos gerados no ciclo de desenvolvimento dos projetos, contando com o suporte de um \textit{software} gratuito e de código aberto para automatizar alguns dos trabalhos manuais, tornando os processos mais ágeis e com armazenamentos seguros.

\subsection{Objetivos Específicos}
\label{sec:objetivos_especificos}

\begin{itemize}
    \item Criar um \textit{software} que auxilie gerentes e líderes a terem reuniões mais objetivas;
    \item implementar mecanismos que permita o controle gerencial das reuniões;
    \item permitir o controle de todas as informações provindas das reuniões;
    \item desenvolver um \textit{software} de código aberto, gratuito e que atenda a demanda de gerenciar as reuniões.
\end{itemize}