\chapter[Introdução]{Introdução}
\label{cp:introducao}

\section{Apresentação do Tema}
\label{sec:apresentacao}

Um dos instrumentos mais fundamentais e que são cada vez mais crescentes na vida organizacional de uma empresa são as reuniões. Segundo \cite{allen2016}, já se gastam até 15\% do tempo coletivo da organização com reuniões. 

Encontros institucionais que não são produtivas e sem sentido, é o que \cite{davidgrady} chama de "Síndrome de Aceitação Sem Sentido"(MAS). David define o MAS como "um reflexo involuntário em que uma pessoa aceita um convite de reunião sem sequer saber o porquê. Uma doença comum entre o escritório e trabalhadores em todo mundo". Atividades que deveriam ser simples e rápidas se tornam complicadas com várias reuniões para a execução completa delas. Reuniões não são nenhum ponto de prazer entre dentro de uma instituição, contudo se os próprios funcionários não conseguem ver o sentido da reunião e os tópicos abordados, mostra que a empresa como um todo está fadada ao fracasso. 

Nesse viés, reuniões é um dos meios usados para programar uma atividade, reunir pessoas em busca de uma solução relacionada ao problema X. Composta por vários encontros sem tópicos e objetivos específicos, sem feedbacks aos gerentes e sem atas sobre o que foi discutido, as reuniões infelizmente se tornam um problema para gerentes e seus funcionários para as instituições. Problemas como estes podem ser pela falta de investimento disponível para aplicação na área tecnológica ou até mesmo pela priorização de outras necessidades.

Este trabalho visa propor uma solução de software para auxiliar a condução de reuniões, utilizando dos conhecimentos adquiridos no curso de \imprimircurso para o desenvolvimento de uma solução que se adeque às reais necessidades empregadas aos gerentes dos projetos organizacionais. 

\section{Justificativa}
\label{sec:justificativa}

Tendo como premissas os problemas em reuniões apresentados no tópico anterior (\ref{sec:apresentacao}), uma das soluções para aumentar a produtividade em reuniões é através de um sistema web que auxilie os gerentes e líderes de reuniões a gerenciar seus encontros de forma rápida, intuitiva e gratuita para que qualquer organização possa usar os recursos do software com o objetivo de melhorar a organização de sua empresa.

O Sistema GRATA, vem oferecer a solução prática para a melhoria do controle das informações e qualidade dos serviços. Tendo como a principal funcionalidade o registro das Atas de reuniões de forma simples e intuitiva, tanto para quem gerencia como para quem participa. Além da automação dos processos essenciais da organização, o sistema fornece relatórios gerenciais e analíticos, que podem ser usados para identificação de pontos de melhoria ou até mesmo para dar visibilidade a questões específicas.

\section{Problema de Pesquisa}
\label{sec:problema_de_pesquisa}

O problema de pesquisa é a pergunta a ser respondida no TCC e que impacta diretamente nos objetivos do projeto. Esse problema direciona também aos objetivos gerais e específicos do projeto. \cite{gomides} define o problema de pesquisa como "algo que você montará para ser solucionado a partir de uma hipótese. A hipótese será uma suposta solução a seu problema, cujo adequação como solução ou não, será a averiguada através de uma pesquisa, usando o problema como uma fórmula para tal".

Nessa viés e utilizando a justificativa do projeto no tópico \ref{sec:justificativa}, é possível se tem o problema de pesquisa. A problemática a ser resolvida neste projeto é: Como desenvolver um sistema para auxiliar a condução de reuniões em organizações?

\subsection{Metodologia}
\label{sec:metodologia_introducao}

A metodologia abordada neste trabalho teve sua escolha baseada após a realização de pesquisas comparativas entre metodologias tradicionais como o \cite{pmbok} e a metodologia ágil de software.

Por conta de um conhecimentos maior sobre a metodologia e com entregas frequentes em menos tempo, a metodologia escolhida para este projeto no desenvolvimento do sistema foi a metodologia ágil, juntamente com algumas práticas do Scrum e baseado nos valores do \textit{Lean Software Development}.  

\subsection{Requisitos}

Requisito não é um termo usado apenas pela \imprimircurso. Há casos em que requisitos são apenas uma declaração abstrata em alto nível de um serviço ou restrição que um sistema deve oferecer.

\cite{sommerville} os define como: "Os requisitos de um sistema são as descrições do que o sistema deve fazer, os serviços que oferece e as restrições a seu funcionamento. Esses requisitos refletem as necessidades dos clientes para um sistema que serve a uma finalidade determinada, como controlar um dispositivo, colocar um pedido ou encontrar informações."

Requisitos podem ser definidos em duas categorias: Requisitos Funcionais e Requisitos Não-Funcionais, ambos serão definidos a seguir.

\subsubsection{Requisitos Funcionais}

Os requisitos funcionais descreve o que o sistema deve de fato ser. Requisitos funcionais podem ser tão específicos quanto necessário,por exemplo, podem ter sistemas com requisitos funcionais gerais e outros que além de refletir os sistemas, também abrangem as formas de trabalho de uma organização. Requisitos funcionais de um sistema deve ser completo, isso quer dizer que todos os serviços requisitados pelo usuário devem ser definidos.

\subsubsection{Requisitos Não-Funcionais}

Requisitos não-funcionais são requisitos que são relacionados as propriedades do sistema como confiabilidade, tempo de espera, desempenho, segurança e até restrições do sistema. Requisitos não-funcionais podem possui tanta relevância quanto os requisitos funcionais, pois em uma reunião de levantamento de requisitos, o cliente sonha o mundo e não está atento se os recursos os próprios recursos e os recursos da emprega conseguem atender ao requisito. Um requisito não-funcional não atendido pode inclusive inutilizar um projeto. Exemplo disso é caso um sistema de uma aeronave não consiga atingir a confiabilidade necessária, não será dado o certificado de segurança para operar, sendo assim a aeronave não poderá voar. 

\subsection{O que é Software?}

\subsection{Processo de Desenvolvimento de Software}

\subsection{Arquitetura de Software}

\subsection{Linguagem de Software}

\section{Objetivos}

\subsection{Objetivos Gerais}

\subsection{Objetivos Específicos}

% \section{Estrutura do TCC}

% Este trabalho possui 7 capítulos: No capítulo \ref{cp:introducao} é apresentada a introdução, no capítulo \ref{cp:fundamentacao} é apresentado o referencial téorico, no capítulo \ref{cp:objetivos} são apresentados os objetivos do projeto, no capítulo \ref{cp:metodologia} é apresentada a metodologia, no capítulo \ref{cp:proposta} é apresentada a proposta da solução, no capítulo \ref{cp:planejamento} é apresentado o planejamento da solução e por último no capítulo \ref{cp:conclusao} onde é mostrada a conclusão do projeto.