\begin{resumo}

O gerenciamento das reuniões é uma área importante no processo de uma empresa, uma vez que é a partir de encontros, sejam presenciais ou não, que os gerentes  desenvolvem seus projetos, estimativam prazos de entregas, e a partir delas podem medir o nível de satisfação de seus funcionários. Conduzir uma reunião é antes de tudo saber que todos os envolvidos no projeto estão cientes dos acontecimentos e possuem pontos de vistas comuns, dando assim a sensação de fazer parte de algo e principalmente fazer com que os envolvidos sintam satisfação com o projeto. O problema gerado com o número elevado de reuniões ineficazes, o aumento na insatisfação dos envolvidos e o custo para gerir uma reunião já são pontos que levantam estudos e estes estudos é um dos pontos que impulsam este trabalho. Diante destes problemas, este trabalho visa desenvolver uma ferramenta que auxilie gerentes na condução de seus projetos e reuniões. 

 \vspace{\onelineskip}

 \noindent
 \textbf{Palavras-chave}: Gerenciamento de Reuniões, Desenvolvimento de \textit{Software}, Satisfação, Produtividade.
\end{resumo}
